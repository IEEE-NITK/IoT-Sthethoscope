\documentclass[conference]{IEEEtran}
\IEEEoverridecommandlockouts
% The preceding line is only needed to identify funding in the first footnote. If that is unneeded, please comment it out.
\usepackage{cite}
\usepackage{amsmath,amssymb,amsfonts}
\usepackage{algorithmic}
\usepackage{graphicx}
\usepackage{textcomp} 


\usepackage{xcolor}
\def\BibTeX{{\rm B\kern-.05em{\sc i\kern-.025em b}\kern-.08em
    T\kern-.1667em\lower.7ex\hbox{E}\kern-.125emX}}
\begin{document}

\title{IoT Stethoscope\\
%{\footnotesize \textsuperscript{*}Note: Sub-titles are not captured in Xplore and
%&should not be used}
\thanks{Identify applicable funding agency here. If none, delete this.}
}

\author{
\IEEEauthorblockN{Jeet Shah}
\IEEEauthorblockA{\textit{dept. of Electrical and Electronics} \\
\textit{National Institute of Tehnology Karnataka}\\
Surathkal,India \\
jeetshah250@gmail.com}
\and
\IEEEauthorblockN{2\textsuperscript{nd} Moha Mankad}
\IEEEauthorblockA{\textit{dept. of Electronics and Communication } \\
\textit{name of organization (of Aff.)}\\
City, Country \\
email address or ORCID}
\and
\IEEEauthorblockN{3\textsuperscript{rd} Given Name Surname}
\IEEEauthorblockA{\textit{dept. name of organization (of Aff.)} \\
\textit{name of organization (of Aff.)}\\
City, Country \\
email address or ORCID}
\and
\IEEEauthorblockN{4\textsuperscript{th} Given Name Surname}
\IEEEauthorblockA{\textit{dept. name of organization (of Aff.)} \\
\textit{name of organization (of Aff.)}\\
City, Country \\
email address or ORCID}
\and
\IEEEauthorblockN{5\textsuperscript{th} Given Name Surname}
\IEEEauthorblockA{\textit{dept. name of organization (of Aff.)} \\
\textit{name of organization (of Aff.)}\\
City, Country \\
email address or ORCID}
\and
\IEEEauthorblockN{6\textsuperscript{th} Given Name Surname}
\IEEEauthorblockA{\textit{dept. name of organization (of Aff.)} \\
\textit{name of organization (of Aff.)}\\
City, Country \\
email address or ORCID}
}

\maketitle

\begin{abstract}
The aim of this project is to develop a smart device which will be able to make a preliminary decision whether or not it is required to visit a Doctor. This device is a handheld wireless stethoscope which will record the heartbeat and lung sounds and send it to a mobile Application which will do the final assessment of the above.The identification and classification will be done with the help of deep learning networks. The project is proposed for arrhythmia and COVID-19. The MIT-BIH database and Coswara database are used to conduct the following research. This device is intended to be used by any person even of novice experience in any sort of medical perspective. 
\end{abstract}

\begin{IEEEkeywords}
Internet of Things, COVID-19, Signal Processing, Stethoscope, Arrhythmia.
\end{IEEEkeywords}

\section{Introduction}
In the recent times we have observed an acute shortage of the doctors in wake of the pandemic. The people are scared and they sometimes visit the doctor even in cases of common cold and other harmless diseases. Hence, there is a need of a device which can do preanalysis of the situation and then predict if there is a need to visit a doctor or not. For this purpose, an IoT stethoscope is proposed in this project. This Stethoscope will Record the heart beat and lung sounds of a person and send them over to the a mobile application. This mobile application is preloaded with a deep learning network which can analyze the sounds. After the analysis, the app will display a judgement on the need for visiting a doctor and also the probability of having the particular disease. This will help to reduce the pressure that is created on the Doctors. The approach which is utilized for doing the project is as follows. The project is divided into 2 parts one is the Hardware part and the other is the Signal Processing part. The Hardware is designed with a mechanical modelling software and WiFi enabled MCUs are used for transmitting the sound to the Mobile application. In the signal processing part the sounds samples are first taken and then converted to a spectrogram which is the representation of sound as an image. These Spectrograms are then fed to a deep learning network. This approach is used because in the recent times Spectrograms have been proved to be very efficient in the identification of certain diseases with very high accuracy. The  continuous wavelet transform has been proved to give an accuracy of 91.8\% in the detection of Atrial Fibrillation [1]. An Short time Fourier transform(STFT) based spectrogram was used along with CNN and has given about 99\% accuracy [2]. Although some of these methods have an excellent accuracy, they haven't been tested with a hardware device which is very important to make it useful for an end user. There have been attempts to make design for intelligent stethoscope for market to detect Heart murmur and make a database for heartbeat sounds[3].
 
\section{Signal Processing}
The signal processing part of the projects deal with the conversion of the signal in to three types of spectrograms namely STFT, MFCCs, and Wavelet Scalogram. Each of these have their particular advantage and their disadvantages. Each of these are discussed in detail ion the following section.

\subsection{Short Time Fourier Transform(STFT)}

The STFTs are based on the Discrete Fourier transforms(DFT). First the signal is windowed using the different windowing function. The most common one and the one that we are using in this project is the Hamming window. Then the DFT is taken for the sections of the signals which are windowed. The frequency response obtained from the DFT is plotted on the y axis where as the x-axis contains the discrete time for which the signals are divided and windowed. The advantages of using STFTs are that they are very fast and can be easily calculated using DFTs. Although some of the signals have a huge variation in amplitudes this might sometimes lead to a problem where a few frequencies remain undistinguished. This is caused by the fixed resolution of the DFT.

\subsection{Wavelet Scalogram}

The Wavelet scalograms are obtained by continuous wavelet transforms. The continuous wavelet transforms are obtained in somewhat a similar fashion to the STFTs. First the wavelet function is scaled using a factor 'a' which is termed as the scale of the transform. The continuous time signal is divided into different scale components using a wavelet function. This wavelet function is then varied along the signal as it progresses in time. Hence here we can see the the Spectrogram can finally be represented by x-axis containing time just like STFT and the y-axis containing the scale a. The biggest advantages that we obtain here the dynamic frequency resolution along the time domain signal. If the value of a is small then we have better resolution and visualization in time. If we have a very high value of a then it will give an excellent frequency resolution. Both of these condition are visible in the spectrogram and hence gives us a better perspective to look at the time domain signal.

\subsection{Mel Frequency Cepstral Coefficient(MFCC)}

Mel frequency is one of the best ways today to visualize the Signal in the form of a spectrogram. It is based on the Mel Scale which plots the frequency according to the sound which is perceived by the human ear. 
\begin{equation}
    m = 2595\log_{10}{(1+\frac{f}{700})}
\end{equation}
Just like the eyes of human being can detect certain colors of light much better than the other colors, ears also follow the same concept and the Mel scale is based on that. To calculate the Coefficients required to make the Mel scale spectrogram first the signal is windowed like the other methods. Then we take the log power at each of the Mel scale frequency. After that we take the Discrete cosine transform(DCT) of the resulting signal of log powers. The amplitudes of the final signals correspond to the MFCCs These coefficients are plotted against their respective time and hence we obtain the spectrogram of the given signal. 

\section{Deep Learning}

The Deep Learning network which is used for this project is the Recurrent Convolution Neural Network (RCNN). The reason we are using the deep learning neural networks are we do not explicitly require to detect a hand crafted feature and feed it into the network. These can conditioned to detect the features themselves and make a prediction. The Convolutional Neural Networks perform very well with images and provide high accuracy. The Recurrent layers of neural networks are perform the task of a memory unit in the network. Although these networks are very convenient to use and have a high accuracy they require quite big data set to be trained. For this project we have the datasets of MIT-BIH and Coswara which are perfect for training the deep learning network.


\subsection{Abbreviations and Acronyms}\label{AA}
Define abbreviations and acronyms the first time they are used in the text, 
even after they have been defined in the abstract. Abbreviations such as 
IEEE, SI, MKS, CGS, ac, dc, and rms do not have to be defined. Do not use 
abbreviations in the title or heads unless they are unavoidable.

\subsection{Units}
\begin{itemize}
\item Use either SI (MKS) or CGS as primary units. (SI units are encouraged.) English units may be used as secondary units (in parentheses). An exception would be the use of English units as identifiers in trade, such as ``3.5-inch disk drive''.
\item Avoid combining SI and CGS units, such as current in amperes and magnetic field in oersteds. This often leads to confusion because equations do not balance dimensionally. If you must use mixed units, clearly state the units for each quantity that you use in an equation.
\item Do not mix complete spellings and abbreviations of units: ``Wb/m\textsuperscript{2}'' or ``webers per square meter'', not ``webers/m\textsuperscript{2}''. Spell out units when they appear in text: ``. . . a few henries'', not ``. . . a few H''.
\item Use a zero before decimal points: ``0.25'', not ``.25''. Use ``cm\textsuperscript{3}'', not ``cc''.)
\end{itemize}

\subsection{Equations}
Number equations consecutively. To make your 
equations more compact, you may use the solidus (~/~), the exp function, or 
appropriate exponents. Italicize Roman symbols for quantities and variables, 
but not Greek symbols. Use a long dash rather than a hyphen for a minus 
sign. Punctuate equations with commas or periods when they are part of a 
sentence, as in:
\begin{equation}
a+b=\gamma\label{eq}
\end{equation}

Be sure that the 
symbols in your equation have been defined before or immediately following 
the equation. Use ``\eqref{eq}'', not ``Eq.~\eqref{eq}'' or ``equation \eqref{eq}'', except at 
the beginning of a sentence: ``Equation \eqref{eq} is . . .''

\subsection{\LaTeX-Specific Advice}

Please use ``soft'' (e.g., \verb|\eqref{Eq}|) cross references instead
of ``hard'' references (e.g., \verb|(1)|). That will make it possible
to combine sections, add equations, or change the order of figures or
citations without having to go through the file line by line.

Please don't use the \verb|{eqnarray}| equation environment. Use
\verb|{align}| or \verb|{IEEEeqnarray}| instead. The \verb|{eqnarray}|
environment leaves unsightly spaces around relation symbols.

Please note that the \verb|{subequations}| environment in {\LaTeX}
will increment the main equation counter even when there are no
equation numbers displayed. If you forget that, you might write an
article in which the equation numbers skip from (17) to (20), causing
the copy editors to wonder if you've discovered a new method of
counting.

{\BibTeX} does not work by magic. It doesn't get the bibliographic
data from thin air but from .bib files. If you use {\BibTeX} to produce a
bibliography you must send the .bib files. 

{\LaTeX} can't read your mind. If you assign the same label to a
subsubsection and a table, you might find that Table I has been cross
referenced as Table IV-B3. 

{\LaTeX} does not have precognitive abilities. If you put a
\verb|\label| command before the command that updates the counter it's
supposed to be using, the label will pick up the last counter to be
cross referenced instead. In particular, a \verb|\label| command
should not go before the caption of a figure or a table.

Do not use \verb|\nonumber| inside the \verb|{array}| environment. It
will not stop equation numbers inside \verb|{array}| (there won't be
any anyway) and it might stop a wanted equation number in the
surrounding equation.

\subsection{Some Common Mistakes}\label{SCM}
\begin{itemize}
\item The word ``data'' is plural, not singular.
\item The subscript for the permeability of vacuum $\mu_{0}$, and other common scientific constants, is zero with subscript formatting, not a lowercase letter ``o''.
\item In American English, commas, semicolons, periods, question and exclamation marks are located within quotation marks only when a complete thought or name is cited, such as a title or full quotation. When quotation marks are used, instead of a bold or italic typeface, to highlight a word or phrase, punctuation should appear outside of the quotation marks. A parenthetical phrase or statement at the end of a sentence is punctuated outside of the closing parenthesis (like this). (A parenthetical sentence is punctuated within the parentheses.)
\item A graph within a graph is an ``inset'', not an ``insert''. The word alternatively is preferred to the word ``alternately'' (unless you really mean something that alternates).
\item Do not use the word ``essentially'' to mean ``approximately'' or ``effectively''.
\item In your paper title, if the words ``that uses'' can accurately replace the word ``using'', capitalize the ``u''; if not, keep using lower-cased.
\item Be aware of the different meanings of the homophones ``affect'' and ``effect'', ``complement'' and ``compliment'', ``discreet'' and ``discrete'', ``principal'' and ``principle''.
\item Do not confuse ``imply'' and ``infer''.
\item The prefix ``non'' is not a word; it should be joined to the word it modifies, usually without a hyphen.
\item There is no period after the ``et'' in the Latin abbreviation ``et al.''.
\item The abbreviation ``i.e.'' means ``that is'', and the abbreviation ``e.g.'' means ``for example''.
\end{itemize}
An excellent style manual for science writers is \cite{b7}.

\subsection{Authors and Affiliations}
\textbf{The class file is designed for, but not limited to, six authors.} A 
minimum of one author is required for all conference articles. Author names 
should be listed starting from left to right and then moving down to the 
next line. This is the author sequence that will be used in future citations 
and by indexing services. Names should not be listed in columns nor group by 
affiliation. Please keep your affiliations as succinct as possible (for 
example, do not differentiate among departments of the same organization).

\subsection{Identify the Headings}
Headings, or heads, are organizational devices that guide the reader through 
your paper. There are two types: component heads and text heads.

Component heads identify the different components of your paper and are not 
topically subordinate to each other. Examples include Acknowledgments and 
References and, for these, the correct style to use is ``Heading 5''. Use 
``figure caption'' for your Figure captions, and ``table head'' for your 
table title. Run-in heads, such as ``Abstract'', will require you to apply a 
style (in this case, italic) in addition to the style provided by the drop 
down menu to differentiate the head from the text.

Text heads organize the topics on a relational, hierarchical basis. For 
example, the paper title is the primary text head because all subsequent 
material relates and elaborates on this one topic. If there are two or more 
sub-topics, the next level head (uppercase Roman numerals) should be used 
and, conversely, if there are not at least two sub-topics, then no subheads 
should be introduced.

\subsection{Figures and Tables}
\paragraph{Positioning Figures and Tables} Place figures and tables at the top and 
bottom of columns. Avoid placing them in the middle of columns. Large 
figures and tables may span across both columns. Figure captions should be 
below the figures; table heads should appear above the tables. Insert 
figures and tables after they are cited in the text. Use the abbreviation 
``Fig.~\ref{fig}'', even at the beginning of a sentence.

\begin{table}[htbp]
\caption{Table Type Styles}
\begin{center}
\begin{tabular}{|c|c|c|c|}
\hline
\textbf{Table}&\multicolumn{3}{|c|}{\textbf{Table Column Head}} \\
\cline{2-4} 
\textbf{Head} & \textbf{\textit{Table column subhead}}& \textbf{\textit{Subhead}}& \textbf{\textit{Subhead}} \\
\hline
copy& More table copy$^{\mathrm{a}}$& &  \\
\hline
\multicolumn{4}{l}{$^{\mathrm{a}}$Sample of a Table footnote.}
\end{tabular}
\label{tab1}
\end{center}
\end{table}

\begin{figure}[htbp]
\centerline{\includegraphics{fig1.png}}
\caption{Example of a figure caption.}
\label{fig}
\end{figure}

Figure Labels: Use 8 point Times New Roman for Figure labels. Use words 
rather than symbols or abbreviations when writing Figure axis labels to 
avoid confusing the reader. As an example, write the quantity 
``Magnetization'', or ``Magnetization, M'', not just ``M''. If including 
units in the label, present them within parentheses. Do not label axes only 
with units. In the example, write ``Magnetization (A/m)'' or ``Magnetization 
\{A[m(1)]\}'', not just ``A/m''. Do not label axes with a ratio of 
quantities and units. For example, write ``Temperature (K)'', not 
``Temperature/K''.

\section*{Acknowledgment}

The preferred spelling of the word ``acknowledgment'' in America is without 
an ``e'' after the ``g''. Avoid the stilted expression ``one of us (R. B. 
G.) thanks $\ldots$''. Instead, try ``R. B. G. thanks$\ldots$''. Put sponsor 
acknowledgments in the unnumbered footnote on the first page.

\section*{References}

Please number citations consecutively within brackets \cite{b1}. The 
sentence punctuation follows the bracket \cite{b2}. Refer simply to the reference 
number, as in \cite{b3}---do not use ``Ref. \cite{b3}'' or ``reference \cite{b3}'' except at 
the beginning of a sentence: ``Reference \cite{b3} was the first $\ldots$''

Number footnotes separately in superscripts. Place the actual footnote at 
the bottom of the column in which it was cited. Do not put footnotes in the 
abstract or reference list. Use letters for table footnotes.

Unless there are six authors or more give all authors' names; do not use 
``et al.''. Papers that have not been published, even if they have been 
submitted for publication, should be cited as ``unpublished'' \cite{b4}. Papers 
that have been accepted for publication should be cited as ``in press'' \cite{b5}. 
Capitalize only the first word in a paper title, except for proper nouns and 
element symbols.

For papers published in translation journals, please give the English 
citation first, followed by the original foreign-language citation \cite{b6}.

\begin{thebibliography}{00}
\bibitem{b1} S. P. Shashikumar, A. J. Shah, Q. Li, G. D. Clifford and S. Nemati, "A deep learning approach to monitoring and detecting atrial fibrillation using wearable technology," 2017 IEEE EMBS International Conference on Biomedical \& Health Informatics (BHI), Orlando, FL, 2017, pp. 141-144, doi: 10.1109/BHI.2017.7897225.
\bibitem{b2} J. Huang, B. Chen, B. Yao and W. He, "ECG Arrhythmia Classification Using STFT-Based Spectrogram and Convolutional Neural Network," in IEEE Access, vol. 7, pp. 92871-92880, 2019, doi: 10.1109/ACCESS.2019.2928017.
\bibitem{b3} F. Yu, A. Bilberg and F. Voss, "The Development of an Intelligent Electronic Stethoscope," 2008 IEEE/ASME International Conference on Mechtronic and Embedded Systems and Applications, Beijing, 2008, pp. 612-617, doi: 10.1109/MESA.2008.4735682.
\bibitem{b5} R. Nicole, ``Title of paper with only first word capitalized,'' J. Name Stand. Abbrev., in press.
\bibitem{b6} Y. Yorozu, M. Hirano, K. Oka, and Y. Tagawa, ``Electron spectroscopy studies on magneto-optical media and plastic substrate interface,'' IEEE Transl. J. Magn. Japan, vol. 2, pp. 740--741, August 1987 [Digests 9th Annual Conf. Magnetics Japan, p. 301, 1982].
\bibitem{b7} M. Young, The Technical Writer's Handbook. Mill Valley, CA: University Science, 1989.
\end{thebibliography}
\vspace{12pt}
\color{red}
IEEE conference templates contain guidance text for composing and formatting conference papers. Please ensure that all template text is removed from your conference paper prior to submission to the conference. Failure to remove the template text from your paper may result in your paper not being published.

\end{document}
